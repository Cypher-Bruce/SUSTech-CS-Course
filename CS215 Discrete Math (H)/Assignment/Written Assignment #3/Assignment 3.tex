\documentclass[a4paper,12pt]{article} 

% First, we usually want to set the margins of our document. For this we use the package geometry.
\usepackage[top = 2.5cm, bottom = 2.5cm, left = 2.5cm, right = 2.5cm]{geometry} 
\usepackage[T1]{fontenc}
\usepackage[utf8]{inputenc}

% The following two packages - multirow and booktabs - are needed to create nice looking tables.
\usepackage{multirow} % Multirow is for tables with multiple rows within one cell.
\usepackage{booktabs} % For even nicer tables.

% As we usually want to include some plots (.pdf files) we need a package for that.
\usepackage{graphicx} 

% The default setting of LaTeX is to indent new paragraphs. This is useful for articles. But not really nice for homework problem sets. The following command sets the indent to 0.
%\usepackage{setspace}
%\setlength{\parindent}{0in}
\usepackage{indentfirst}

% Package to place figures where you want them.
\usepackage{float}

% The fancyhdr package let's us create nice headers.
\usepackage{fancyhdr}

\usepackage{amsmath,amsthm,amsfonts}
\newtheorem{lemma}{Lemma}

% To make our document nice we want a header and number the pages in the footer.

\pagestyle{fancy} % With this command we can customize the header style.

\fancyhf{} % This makes sure we do not have other information in our header or footer.

\lhead{\footnotesize Discrete Mathematics(H): Homework 3}% \lhead puts text in the top left corner. \footnotesize sets our font to a smaller size.

%\rhead works just like \lhead (you can also use \chead)
\rhead{\footnotesize Mengxuan Wu} %<---- Fill in your lastnames.

% Similar commands work for the footer (\lfoot, \cfoot and \rfoot).
% We want to put our page number in the center.
\cfoot{\footnotesize \thepage} 

\begin{document}

\thispagestyle{empty} % This command disables the header on the first page. 

\begin{tabular}{p{15.5cm}}
{\large \bf Discrete Mathematics(H)} \\
Southern University of Science and Technology \\ Mengxuan Wu \\ 12212006 \\
\hline
\\
\end{tabular}

\vspace*{0.3cm} %add some vertical space in between the line and our title.

\begin{center}
	{\Large \bf Assignment 3}
	\vspace{2mm}

	{\bf Mengxuan Wu}
		
\end{center}  

\vspace{0.4cm}

\section*{Q.1}

\subsection*{(a)}

\begin{align*}
	12! =& 12 \times 11 \times 10 \times 9 \times 8 \times 7 \times 6 \times 5 \times 4 \times 3 \times 2 \times 1 \\
	=& (2^2 \times 3) \times 11 \times (2 \times 5) \times 3^2 \times 2^3 \times 7 \times (2 \times 3) \times 5 \times 2^2 \times 3 \times 2 \times 1 \\ 
	=& 2^{10} \times 3^5 \times 5^2 \times 7 \times 11
\end{align*}

\subsection*{(b)}

\begin{align*}
	6560 =& 2 \times 3280 \\
	=& 2^2 \times 1640 \\
	=& 2^3 \times 820 \\
	=& 2^4 \times 410 \\
	=& 2^5 \times 205 \\
	=& 2^5 \times 5 \times 41 
\end{align*}

\section*{Q.2}

\subsection*{(a)}

\begin{align*}
	312 =& 2 \times 156 \\
	=& 2^2 \times 78 \\
	=& 2^3 \times 39 \\
	=& 2^3 \times 3 \times 13 
\end{align*}

\subsection*{(b)}

\begin{align*}
	312 \div 97 =& 3 ... 21 \\
	97 \div 21 =& 4 ... 13 \\
	21 \div 13 =& 1 ... 8 \\
	13 \div 8 =& 1 ... 5 \\
	8 \div 5 =& 1 ... 3 \\
	5 \div 3 =& 1 ... 2 \\
	3 \div 2 =& 1 ... 1 \\
	2 \div 1 =& 2 ... 0
\end{align*}

Therefore, $\gcd(312,97) = 1$.

\subsection*{(c)}

\begin{alignat*}{4}
	1 =& 3 - 1 \times 2 & &  \\
	=& 3 - 1 \times (5 - 3) &=& 2 \times 3 - 1 \times 5 \\
	=& 2 \times (8 - 5) - 1 \times 5 &=& 2 \times 8 - 3 \times 5 \\
	=& 2 \times 8 - 3 \times (13 - 8) &=& 5 \times 8 - 3 \times 13 \\
	=& 5 \times (21 - 13) - 3 \times 13 &=& 5 \times 21 - 8 \times 13 \\
	=& 5 \times 21 - 8 \times (97 - 4 \times 21) &=& 37 \times 21 - 8 \times 97 \\
	=& 37 \times (312 - 3 \times 97) - 8 \times 97 &=& 37 \times 312 - 119 \times 97
\end{alignat*}

Therefore, $1 = 37 \times 312 - 119 \times 97$.
Equivalently, $s = 37$ and $t = 119$ are the solutions to $312s + 97t = \gcd(312,97)$.

\subsection*{(d)}

\begin{alignat*}{3}
	312x &\equiv 3 &\pmod{97} \\
	37 \cdot 312x &\equiv 37 \cdot 3 &\pmod{97} \\
	x &\equiv 111 &\pmod{97} \\
	x &\equiv 14 &\pmod{97}
\end{alignat*}

\section*{Q.3}

Let $d = \gcd(b+a,b-a)$. 
By definition, $d \mid (b+a)$ and $d \mid (b-a)$.
Therefore, $d \mid (b+a) + (b-a) = 2b$ and $d \mid (b+a) - (b-a) = 2a$.
Since $d \mid 2b$ and $d \mid 2a$, $d \mid \gcd(2b,2a) = 2\gcd(b,a) = 2$.

Hence, $d = 1$ or $d = 2$.
Equivalently, we can say that $\gcd(b+a,b-a) \leq 2$.

\section*{Q.4}

\begin{proof}
$ $

For any $x, y$ that $x = y$ and $x, y \in \mathbb{Z}^+$, we can infer that $222 \mid 2^y - 2^x = 0$.
\end{proof}

\section*{Q.5}

\subsection*{(a)}

Yes.

First, we can factorize $561$ into $3 \times 11 \times 17$.
By Fermat's Little Theorem, we have:

\begin{alignat*}{3}
	2^{2}  &\equiv 1 \pmod{3}& \\ 
	2^{10} &\equiv 1 \pmod{11}& \\
	2^{16} &\equiv 1 \pmod{17}&
\end{alignat*}

Therefore, we can find:
\begin{alignat*}{4}
	2^{560} &\equiv 2^{2 \times 280} &\equiv 1 &\pmod{3} \\
	2^{560} &\equiv 2^{10 \times 56} &\equiv 1 &\pmod{11} \\
	2^{560} &\equiv 2^{16 \times 35} &\equiv 1 &\pmod{17}
\end{alignat*}

Hence, $2^{560} \equiv 1 \pmod{561}$.

\subsection*{(b)}

No.

$561$ is not a prime number, since $561 = 3 \times 11 \times 17$.

\section*{Q.6}

\begin{proof}
$ $

\textbf{Sufficient Condition:}

Assume, without loss of generality, that $b \geq a$.
Let $x = \gcd(a,b)$, $y = \text{lcm}(a,b)$. 

By definition, $xy = ab$. 
Since $x + y = a + b$ and $a, b$ are positive integers, we can infer that:
\begin{align*}
	(x + y)^2 =& (a + b)^2 \\
	(x + y)^2 - 4xy =& (a + b)^2 - 4ab \\
	(x - y)^2 =& (a - b)^2 \\
	y - x =& b - a \\
	y - x + x + y =& b - a + a + b \\
	2y =& 2b \\
	y =& b
\end{align*}

Therefore, $y = b$ and $x = a$.
Since $\gcd(a,b) = a$, we can infer that $a \mid b$.

\textbf{Necessary Condition:}

Assume, without loss of generality, that $b \geq a$.

Since $a \mid b$, it is obvious that $\gcd(a,b) = a$ and $\text{lcm}(a,b) = b$.
Therefore, $\gcd(a,b) + \text{lcm}(a,b) = a + b$.
\end{proof}

\section*{Q.7}

\subsection*{(1)}

\begin{proof}[Proof by Cases]
$ $

\textbf{Case 1:} $x$ is an even number.

Since $x$ is an even number, $x^2$ is also an even number.
Therefore, $x^2 - 31$ is an odd number and is not divisible by $36$.

Hence, $x^2 \not\equiv 31 \pmod{36}$.

\textbf{Case 2:} $x$ is an odd number.

Let $x = 2k + 1$ where $k \in \mathbb{Z}$.
Then, $x^2 = 4k^2 + 4k + 1 = 4k(k + 1) + 1$.

Since $4$ is a factor of $36$, we can infer that $x^2 \equiv 31 \pmod{4}$ should also be true.
However, $x^2 \equiv 4k(k + 1) + 1 \equiv 1 \pmod{4}$ and $31 \equiv 3 \pmod{4}$, which is a contradiction.
\end{proof}

\subsection*{(2)}

We can only find two solutions for each of these equations:
\begin{equation*}
	\begin{cases}
		x \equiv 14 \text{ or } 17 &\pmod{31} \\
		x \equiv 17 \text{ or } 20 &\pmod{37} \\
	\end{cases}
\end{equation*}

By Chinese Remainder Theorem, we can find four solutions for this system of linear congruences:
\begin{equation*}
	x \equiv 17 \text{ or } 572 \text{ or } 575 \text{ or } 1130 \pmod{1147}
\end{equation*}
\section*{Q.8}

\begin{proof}[Proof by Contradiction]
$ $

\begin{lemma}
	For any positive integers $a,m$ such that $\gcd(a,m) \ne 1$, there exists a positive integer $b$ where $b \in \mathbb{Z}_m$ such that $ab \equiv 0 \pmod{m}$.

	\begin{proof}
	$ $

	Let $d = \gcd(a,m)$.
	By definition, $d \mid a$ and $d \mid m$.
	Assume that $a = kd$ and $m = ld$ where $k,l \in \mathbb{Z}$.
	It's obvious that $l \in \mathbb{Z}_m$ and $l \ne 0$, since $l = \frac{m}{d}$ and $d > 1$.

	Since $la \equiv lkd \equiv km \equiv 0 \pmod{m}$, we can infer that $b = l$ is the positive integer we are looking for.
	\end{proof}
\end{lemma}

By the lemma above, we can always find a positive integer $b$ where $b \in \mathbb{Z}_m$ such that $ab \equiv 0 \pmod{m}$.

If $a$ has an inverse $\bar{a}$ modulo $m$, then we have:
\begin{alignat*}{3}
	a \bar{a} &\equiv 1 &\pmod{m}\\
	ab \bar{a} &\equiv b &\pmod{m}\\
	0 \bar{a} &\equiv b &\pmod{m}\\
	0 &\equiv b &\pmod{m}
\end{alignat*}

This is a contradiction, since $b \ne 0$ and $b \in \mathbb{Z}_m$.
\end{proof}

\section*{Q.9}

\subsection*{(a)}

\begin{align*}
	321 \div 2 =& 160 ... 1 \\
	160 \div 2 =& 80 ... 0 \\
	80 \div 2 =& 40 ... 0 \\
	40 \div 2 =& 20 ... 0 \\
	20 \div 2 =& 10 ... 0 \\
	10 \div 2 =& 5 ... 0 \\
	5 \div 2 =& 2 ... 1 \\
	2 \div 2 =& 1 ... 0 \\
	1 \div 2 =& 0 ... 1
\end{align*}

Therefore, $321_{10} = 101000001_2$.

\subsection*{(b)}

\begin{align*}
	1023 =& 2^{10} - 1\\
	=& (10000000000 - 1)_2\\
	=& 1111111111_2
\end{align*}

Therefore, $1023_{10} = 1111111111_2$.

\subsection*{(c)}

\begin{align*}
	100632 \div 2 =& 50316 ... 0 \\
	50316 \div 2 =& 25158 ... 0 \\
	25158 \div 2 =& 12579 ... 0 \\
	12579 \div 2 =& 6289 ... 1 \\
	6289 \div 2 =& 3144 ... 1 \\
	3144 \div 2 =& 1572 ... 0 \\
	1572 \div 2 =& 786 ... 0 \\
	786 \div 2 =& 393 ... 0 \\
	393 \div 2 =& 196 ... 1 \\
	196 \div 2 =& 98 ... 0 \\
	98 \div 2 =& 49 ... 0 \\
	49 \div 2 =& 24 ... 1 \\
	24 \div 2 =& 12 ... 0 \\
	12 \div 2 =& 6 ... 0 \\
	6 \div 2 =& 3 ... 0 \\
	3 \div 2 =& 1 ... 1 \\
	1 \div 2 =& 0 ... 1
\end{align*}

Therefore, $100632_{10} = 11000100100011000_2$.

\section*{Q.10}

Using Bezout's Theorem, there exists integers $s_n, t_n$ such that:
\begin{align*}
	s_1 p + t_1 q =& \gcd(p,q) = 1 \\
	s_2 p + t_2 r =& \gcd(p,r) = 1 \\
	s_3 q + t_3 r =& \gcd(q,r) = 1
\end{align*}

By multiply these terms together, we have:
\begin{align*}
	(s_1 p + t_1 q) (s_2 p + t_2 r) (s_3 q + t_3 r) =& s_1 s_2 s_3 p^2q + s_1 s_2 t_3 p^2r + s_1 t_2 s_3 pqr + s_1 t_2 t_3 pr^2 \\
	 &+ t_1 s_2 s_3 pq^2 + t_1 s_2 t_3 pqr + t_1 t_2 s_3 q^2r + t_1 t_2 t_3 qr^2 \\
	=& (s_1 s_2 s_3 p + t_1 s_2 s_3 q + s_1 t_2 s_3 r + t_1 s_2 t_3 r) pq \\ 
	 &+ (t_1 t_2 s_3 q + t_1 t_2 t_3 r) qr + (s_1 s_2 t_3 p + s_1 t_2 t_3 r) rp \\
	=& 1
\end{align*}

Therefore, we find $a = s_1 s_2 s_3 p + t_1 s_2 s_3 q + s_1 t_2 s_3 r + t_1 s_2 t_3 r$, $b = t_1 t_2 s_3 q + t_1 t_2 t_3 r$ and $c = s_1 s_2 t_3 p + s_1 t_2 t_3 r$ that satisfy $a(pq) + b(qr) + c(rp) = 1$.

\section*{Q.11}

By Fermat's Little Theorem, we have $10^{12} \equiv 1 \pmod{13}$.
Therefore, we can infer that:

\begin{equation*}
	10^{100} \equiv 10^{12 \times 8 + 4} \equiv 10^4 \equiv 3 \pmod{13}
\end{equation*}

Since $3^3 \equiv 27 \equiv 1 \pmod{13}$ and $3 \mid 10^{100} - 1$, we can infer that:
\begin{equation*}
	(10^{100})^{(10^{100})} \equiv 3^{(10^{100})} \equiv 3^1 \equiv 3 \pmod{13}
\end{equation*}

Hence, $(10^{100})^{(10^{100})} \equiv 3 \pmod{13}$.

\section*{Q.12}

\subsection*{(1)}

\begin{proof}
$ $

\begin{align*}
	f(cm) =& c + a_1 cm + a_2 c^2 m^2 + a_3 c^3 m^3 + ... + a_{n-1} c^{n-1} m^{n-1} + c^n m^n \\
	=& c (1 + a_1 m + a_2 c m^2 + a_3 c^2 m^3 + ... + a_{n-1} c^{n-2} m^{n-1} + c^{n-1} m^n)
\end{align*}

Therefore, $f(cm)$ is a multiple of $c$.
\end{proof}

\subsection*{(2)}

\begin{proof}
$ $

We only consider the case when $n = cm$ where $m \in \mathbb{Z}$.
Since $f(n)$ grows unboundedly to infinity, we can expect an $m_0$ that for all $m \geq m_0$, $f(cm) > c$.

From the proof above, we can infer that $f(cm)$ is a multiple of $c$.
Therefore, $\frac{f(cm)}{c}$ is a factor of $f(cm)$ and $\frac{f(cm)}{c} > 1$.
Since $f(cm) > c > 1$, $f(cm)$ is a composite number.

We can find infinitely many $m$ that $m \geq m_0$.
Therefore, there exists infinitely many $f(cm)$ that is not a prime number.
\end{proof}

\subsection*{(3)}

\begin{proof}
$ $

From the proof above, we can infer that when $c > 1$, there exists infinitely many $n$ that $f(n)$ is not a prime number.
When $c \leq 1$, $f(0) = c \leq 1$ and it will not be a prime number.

In conclusion, non-constant polynomial $f(n)$ cannot generate only prime numbers for all $n \in \mathbb{N}$.
\end{proof}

\newpage
\section*{Q.13}

\begin{proof}[Proof by Contradiction]
$ $

By definition, we know $2^{\log_2 3} = 3$.

If $\log_2 3$ is a rational number, then we can write $\log_2 3 = \frac{a}{b}$ where $a,b \in \mathbb{Z}$ and $b \ne 0$.
Since $\log_2 3 > 0$, without loss of generality, we can assume that $a > 0$ and $b > 0$.

Therefore, we have:
\begin{align*}
	2^{\frac{a}{b}} =& 3 \\
	2^a =& 3^b
\end{align*}

This is a contradiction, since $2^a$ is an even number and $3^b$ is an odd number.
\end{proof}

\section*{Q.14}

\begin{proof}
$ $

Assume $\bar{a}_1, \bar{a}_2$ are two inverse of $a$ modulo $m$.
Then, we have:
\begin{alignat*}{3}
	\bar{a}_1 a &\equiv 1 &\pmod{m} \\
	\bar{a}_2 a &\equiv 1 &\pmod{m} \\
	(\bar{a}_1 - \bar{a}_2) a &\equiv 0 &\pmod{m}
\end{alignat*}


Equivalently, we have $m \mid (\bar{a}_1 - \bar{a}_2) a$.

	\begin{lemma}
	$ $

	If $a$, $b$, $c$ are positive integers such that $\gcd(a, b) = 1$ and $a \mid bc$, then $a \mid c$.
	\end{lemma}

Since $a$ and $m$ are relatively prime and $m \mid (\bar{a}_1 - \bar{a}_2) a$, we can infer that $m \mid (\bar{a}_1 - \bar{a}_2)$.
Equivalently, we have $\bar{a}_1 \equiv \bar{a}_2 \pmod{m}$.
Hence, the inverse of $a$ modulo $m$ is unique modulo $m$.
\end{proof}

\section*{Q.15}

\begin{proof}[Proof by Contradiction]
$ $

Suppose that there are only finitely many primes of the form $4k+3$ where $k \in \mathbb{N}$.
Let them be $q_1, q_2, ..., q_n$.
Obviously, $4q_1q_2 \cdots q_n - 1 \equiv -1 \equiv 3 \pmod{4}$

Firstly, $2$ is not a factor of $4q_1q_2 \cdots q_n - 1$, since $4q_1q_2 \cdots q_n - 1 \equiv 3 \pmod{4}$, which means it is an odd number.

Secondly, $q_i$ is not a factor of $4q_1q_2 \cdots q_n - 1$, since $4q_1q_2 \cdots q_n - 1 \equiv -1 \pmod{q_i}$ where $i \in \{1,2,...,n\}$.

Thirdly, prime factors of $4q_1q_2 \cdots q_n - 1$ cannot all be of the form $4k+1$, since that:
\begin{equation*}
	(4k_1 + 1)^{c_1} (4k_2 + 1)^{c_2} (4k_3 + 1)^{c_3} \cdots \equiv 1 \not\equiv 3 \equiv 4q_1q_2 \cdots q_n - 1 \pmod{4}
\end{equation*}

Since all prime number except $2$ can be written as $4k+1$ or $4k+3$, we can infer that $4q_1q_2 \cdots q_n - 1$ must have a prime factor of the form $4k+3$ and is not in the list $q_1, q_2, ..., q_n$.
\end{proof}

\section*{Q.16}

\subsection*{(a)}

Using Fermat's Little Theorem, we have:
\begin{alignat*}{6}
	5^{2003} \equiv& 5^{333 \times 6 + 5} &\equiv& 5^5 &\equiv& 3 \pmod{7}& \\
	5^{2003} \equiv& 5^{200 \times 10 + 3} &\equiv& 5^3 &\equiv& 4 \pmod{11}& \\
	5^{2003} \equiv& 5^{166 \times 12 + 11} &\equiv& 5^{11} &\equiv& 8 \pmod{13}&
\end{alignat*}

\subsection*{(b)}

Using Chinese Remainder Theorem, we can find:
\begin{align*}
	M_1 =& 11 \times 13 = 143 \\
	M_2 =& 7 \times 13 = 91 \\
	M_3 =& 7 \times 11 = 77
\end{align*}

Using Extended Euclidean Algorithm, we can find their inverses:
\begin{alignat*}{3}
	5 \times 143 \equiv& 1 \pmod{7}& \\
	4 \times 91 \equiv& 1 \pmod{11}& \\
	12 \times 77 \equiv& 1 \pmod{13}&
\end{alignat*}

Therefore, we have:
\begin{alignat*}{3}
	5^{2003} \equiv& 3 \times 5 \times 143 + 4 \times 4 \times 91 + 8 \times 12 \times 77 &\pmod{1001} \\
	\equiv& 10993 &\pmod{1001} \\
	\equiv& 983 &\pmod{1001}
\end{alignat*}

\section*{Q.17}

\begin{proof}
$ $

If $a \equiv b \pmod{m_i}$ for $i = 1,2,...,n$ and $m_i$ are pairwise relatively prime, then we have:
\begin{align*}
	a \equiv& b \pmod{m_1} \\
	a \equiv& b \pmod{m_2} \\
	... \\
	a \equiv& b \pmod{m_n}
\end{align*} 

By definition, we know that $m_1 \mid (a - b)$, $m_2 \mid (a - b)$, ..., $m_n \mid (a - b)$.

	\begin{lemma}
	$ $

	If $a$, $b$, $c$ are positive integers such that $a \mid c$ and $b \mid c$, then $\text{lcm}(a,b) \mid c$.

		\begin{proof}
		$ $

		Consider the factorization of $a$ and $b$, we assume that $a = p_1^{c_1} p_2^{c_2} \cdots p_n^{c_n}$ and $b = p_1^{d_1} p_2^{d_2} \cdots p_n^{d_n}$ where $p_i$ are prime numbers and $c_i, d_i \in \mathbb{Z}$.
		Without loss of generality, we can assume that $c_i + d_i > 0$ for all $i$.

		For every prime $p_i$, we can infer that $p_i^{c_i} \mid c$ and $p_i^{d_i} \mid c$.
		Therefore, $p_i^{\text{max}(c_i,d_i)} \mid c$.
		By definition, we know that $\text{lcm}(a,b) = p_1^{\text{max}(c_1,d_1)} p_2^{\text{max}(c_2,d_2)} \cdots p_n^{\text{max}(c_n,d_n)} \mid c$.
		\end{proof}
	\end{lemma}

By the lemma above, we can infer that $\text{lcm}(m_1,m_2,...,m_n) \mid (a - b)$.
Since $m_1, m_2, ..., m_n$ are pairwise relatively prime, we can infer that $\text{lcm}(m_1,m_2,...,m_n) = m_1 m_2 \cdots m_n = m$.
Therefore, $m \mid (a - b)$.
Equivalently, we have $a \equiv b \pmod{m}$.
\end{proof}

\section*{Q.18}

\begin{proof}
$ $

If there exist $a,b$ that are both solution to a system of linear congruences modulo pair wise relatively prime moduli $m_1,m_2,...,m_n$, then we have:
\begin{alignat*}{4}
	a &\equiv b &\equiv c_1 &\pmod{m_1} \\
	a &\equiv b &\equiv c_2 &\pmod{m_2} \\
	... \\
	a &\equiv b &\equiv c_n &\pmod{m_n}
\end{alignat*}

By the proof of Q.17, we can infer that $a \equiv b \pmod{m}$ where $m = m_1 m_2 \cdots m_n$.
Equivalently, we say the solution is unique modulo $m$.
\end{proof}

\section*{Q.19}

Since these moduli are not pair wise relatively prime, we factorize them into prime numbers.
The given conditions can be factorized into:
\begin{alignat*}{3}
	x &\equiv 1 &\pmod{2} \\
	x &\equiv 2 &\pmod{3} \\
	x &\equiv 3 &\pmod{5} 
\end{alignat*}

Using Chinese Remainder Theorem, we can find:
\begin{align*}
	M_1 =& 3 \times 5 = 15 \\
	M_2 =& 2 \times 5 = 10 \\
	M_3 =& 2 \times 3 = 6
\end{align*}

Using Extended Euclidean Algorithm, we can find their inverses:
\begin{alignat*}{3}
	1 \times 15 \equiv& 1 \pmod{2}& \\
	1 \times 10 \equiv& 1 \pmod{3}& \\
	1 \times 6 \equiv& 1 \pmod{5}&
\end{alignat*}

Therefore, we have:
\begin{alignat*}{3}
	x \equiv& 1 \times 1 \times 15 + 2 \times 1 \times 10 + 3 \times 1 \times 6 &\pmod{30} \\
	\equiv& 53 &\pmod{30} \\
	\equiv& 23 &\pmod{30}
\end{alignat*}

The solution is of the form $x = 23 + 30k$ where $k \in \mathbb{Z}$.

\section*{Q.20}

These given conditions can be written as:
\begin{alignat*}{3}
	4 &\equiv (7a+c) &\pmod{11} \\
	6 &\equiv (4a+c) &\pmod{11} \\
\end{alignat*}

By subtracting the second equation from the first equation, we have:
\begin{alignat*}{3}
	3a &\equiv -2 &\pmod{11} \\
	4 \cdot 3a &\equiv 4 \cdot -2 &\pmod{11} \\
	a &\equiv -8 &\pmod{11} \\
	a &\equiv 3 &\pmod{11}
\end{alignat*}

Substitute $a = 3$ into the first equation, we have:
\begin{alignat*}{3}
	21 + c &\equiv 4 &\pmod{11} \\
	c &\equiv -17 &\pmod{11} \\
	c &\equiv 5 &\pmod{11}
\end{alignat*}

Hence, the next number is $6 \times 3 + 5 \mod 11 = 1$.

\section*{Q.21}

\begin{proof}
$ $

\textbf{Proof of $\phi(m) \mid \phi(n)$:}

Assume the factorization of $m$ is that $m = p_1^{c_1} p_2^{c_2} \cdots p_n^{c_n}$ where $p_i$ are prime numbers and $c_i \in \mathbb{Z}$.
Furthermore, we assume the factorization of $n$ is that $n = p_1^{d_1} p_2^{d_2} \cdots p_n^{d_n} q_1^{e_1} q_2^{e_2} \cdots q_m^{e_m}$ where $q_i$ are prime numbers and $d_i, e_i \in \mathbb{Z}$.
Without loss of generality, we can assume that $c_i, d_i, e_i > 0$ for all $i$ and $c_i \leq d_i$.

Since $p_1^{d_1} p_2^{d_2} \cdots p_n^{d_n}$ is relatively prime to $q_1^{e_1} q_2^{e_2} \cdots q_m^{e_m}$, we can infer that:
\begin{align*}
	\phi(n) =& \phi(p_1^{d_1} p_2^{d_2} \cdots p_n^{d_n}) \phi(q_1^{e_1} q_2^{e_2} \cdots q_m^{e_m})\\
	=& \phi(p_1^{d_1}) \phi(p_2^{d_2}) \cdots \phi(p_n^{d_n}) \phi(q_1^{e_1} q_2^{e_2} \cdots q_m^{e_m})
\end{align*}

Then, for every prime factor $p_i$ of $m$, we have:
\begin{align*}
	\phi(p_i^{d_i}) =& p_i^{d_i} - p_i^{d_i - 1} \\
	=& (p_i^{c_i} - p_i^{c_i - 1}) p_i^{d_i - c_i} \\
	=& \phi(p_i^{c_i}) p_i^{d_i - c_i}
\end{align*}

Since $c_i \leq d_i$, we can infer that $p_i^{d_i - c_i} \geq 1$.
Therefore, $\phi(p_i^{c_i}) \mid \phi(p_i^{d_i})$.
Combining all terms, we have $\phi(m) \mid \phi(n)$.

\newpage
\textbf{Proof of $\phi(mn) = m \phi(n)$:}

For any prime $p$ and positive integers $c,d$, we have:

\begin{align*}
	\phi(p^{c+d}) &= p^{c+d} - p^{c+d-1} \\
	&= p^c (p^d - p^{d-1}) \\
	&= p^c \phi(p^d)
\end{align*}

For $\phi(mn)$, we can infer that:
\begin{align*}
	\phi(mn) =& \phi(p_1^{c_1+d_1} p_2^{c_2+d_2} \cdots p_n^{c_n+d_n} q_1^{e_1} q_2^{e_2} \cdots q_m^{e_m}) \\
	=& \phi(p_1^{c_1+d_1} p_2^{c_2+d_2} \cdots p_n^{c_n+d_n}) \phi(q_1^{e_1} q_2^{e_2} \cdots q_m^{e_m}) \\
	=& \phi(p_1^{c_1+d_1}) \phi(p_2^{c_2+d_2}) \cdots \phi(p_n^{c_n+d_n}) \phi(q_1^{e_1} q_2^{e_2} \cdots q_m^{e_m}) \\
	=& p_1^{c_1} \phi(p_1^{d_1}) p_2^{c_2} \phi(p_2^{d_2}) \cdots p_n^{c_n} \phi(p_n^{d_n}) \phi(q_1^{e_1} q_2^{e_2} \cdots q_m^{e_m}) \\
	=& [p_1^{c_1} p_2^{c_2} \cdots p_n^{c_n}] [\phi(p_1^{d_1}) \phi(p_2^{d_2}) \cdots \phi(p_n^{d_n}) \phi(q_1^{e_1} q_2^{e_2} \cdots q_m^{e_m})] \\
	=& m \phi(n)
\end{align*}

Therefore, $\phi(mn) = m \phi(n)$.
\end{proof}

\section*{Q.22}

\begin{proof}
$ $

Since we know $n = pq$ and the value of $(p-1)(q-1)$, then we can find $p+q$ by solving the following equation:
\begin{align*}
	(p-1)(q-1) =& pq - p - q + 1 \\
	(p-1)(q-1) =& pq - (p + q) + 1 \\
	p + q =& pq - (p-1)(q-1) + 1 
\end{align*}

Let $s = p + q$, then we have:
\begin{align*}
	p^2 -ps + pq =& p^2 - p(p+q) + pq \\
	=& p^2 -p^2 - pq + pq \\
	=& 0
\end{align*}

Therefore, $p$ is a root of the equation $p^2 - ps + pq = 0$.
Then we can find $p$ by solving the quadratic equation.
Equivalently, we have:
\begin{equation*}
	p = \frac{s \pm \sqrt{s^2 - 4n}}{2}
\end{equation*}

Then, we can find $q$ by $q = \frac{n}{p}$.

This equation always has two real roots, this is because:
\begin{align*}
	s^2 - 4n =& (p + q)^2 - 4pq \\
	=& p^2 + 2pq + q^2 - 4pq \\
	=& p^2 - 2pq + q^2 \\
	=& (p - q)^2 \\
	\geq& 0
\end{align*}

And both roots are always positive, since:
\begin{equation*}
	s - \sqrt{s^2 - 4n} = \sqrt{s^2} - \sqrt{s^2 - 4n} > 0
\end{equation*}
\end{proof}

\section*{Q.23}

\subsection*{(a)}

\begin{equation*}
	\hat{M} = M^e \bmod n = 8^7 \bmod 65 = 57
\end{equation*}

\subsection*{(b)}

Since $n = 65 = 5 \times 13$, we can find $p = 5$ and $q = 13$.
Then, we can find $\phi(n) = (p-1)(q-1) = 4 \times 12 = 48$.

The private key $d$ then will be the inverse of $e$ modulo $\phi(n)$.
Using Extended Euclidean Algorithm, we can find $7 \times 7 \equiv 1 \pmod{48}$.
Therefore, $d = 7$.

\subsection*{(c)}

\begin{equation*}
	M = \hat{M}^d \bmod n = 57^7 \bmod 65 = 8
\end{equation*}

\section*{Q.24}

\begin{proof}[Proof by Cases]
$ $

Since $(p-1)(q-1) = \gcd(p-1,q-1) \cdot \text{lcm}(p-1,q-1)$, we can infer that $\lambda(n) \mid \phi(n)$.
By definition, $\gcd(e,\phi(n)) = 1$, and then we know $\gcd(e,\lambda(n)) = 1$.
Hence, we can always find $d'$ such that $ed' \equiv 1 \pmod{\lambda(n)}$.

\textbf{Case 1:} $\gcd(M,n) = 1$

Since $ed' \equiv 1 \pmod{\lambda(n)}$, we can assume that $ed' - 1 = k \lambda(n)$ where $k \in \mathbb{Z}$.
Also, we can assume $\lambda(n) = t (p - 1) = s (q - 1)$ where $t,s \in \mathbb{Z}$.
Then, we have: 

\begin{alignat*}{3}
	C^{d'} &\equiv M^{ed'} &\pmod{p} \\
	&\equiv M^{k \lambda(n)} \cdot M &\pmod{p} \\
	&\equiv M^{k t (p - 1)} \cdot M &\pmod{p} \\
	&\equiv (M^{p-1})^{k t} \cdot M &\pmod{p}
\end{alignat*}

Since $\gcd(M,n) = 1$ and $n = pq$, we know that $\gcd(M,p) = 1$.
By Fermat's Little Theorem, we know that $M^{p-1} \equiv 1 \pmod{p}$.
Therefore, we have:

\begin{alignat*}{3}
	C^{d'} &\equiv (M^{p-1})^{k t} \cdot M &\pmod{p}\\
	&\equiv 1^{k t} \cdot M &\pmod{p}\\
	&\equiv M &\pmod{p}
\end{alignat*}

Similarly, we can infer that $C^{d'} \equiv M \pmod{q}$.
Since $p$ and $q$ are relatively prime, we can infer that $C^{d'} \equiv M \pmod{n}$.

\textbf{Case 2:} $\gcd(M,n) = p$

To proof $C^{d'} \equiv M^{ed'} \equiv M \pmod{n}$ is equivalent to proof $n \mid M (M^{ed' - 1} - 1)$. 
Since $p \mid M$ and $n = pq$, we only need to proof $q \mid M^{ed' - 1} - 1$.
Equivalently, we need to proof $M^{ed' - 1} \equiv 1 \pmod{q}$.

Since $ed' \equiv 1 \pmod{\lambda(n)}$, we can assume that $ed' - 1 = k \lambda(n)$ where $k \in \mathbb{Z}$.
Also, we can assume $\lambda(n) = t (p - 1) = s (q - 1)$ where $t,s \in \mathbb{Z}$.
Then, we have: 

\begin{alignat*}{3}
	M^{ed' - 1} &\equiv M^{k \lambda(n)} &\pmod{q} \\
	&\equiv M^{k s (q - 1)} &\pmod{q} \\
	&\equiv (M^{q-1})^{k s} &\pmod{q}
\end{alignat*}

Since $\gcd(M,n) = p$ and $n = pq$, we know that $\gcd(M,q) = 1$ still holds.
By Fermat's Little Theorem, we know that $M^{q-1} \equiv 1 \pmod{q}$.

\begin{alignat*}{3}
	M^{ed' - 1} &\equiv (M^{q-1})^{k s} &\pmod{q} \\
	&\equiv 1^{k s} &\pmod{q} \\
	&\equiv 1 &\pmod{q}
\end{alignat*}

Hence, $C^{d'} \equiv M \pmod{n}$.

\textbf{Case 3:} $\gcd(M,n) = q$

Similar to Case 2, we can proof $C^{d'} \equiv M \pmod{n}$.

\textbf{Case 4:} $\gcd(M,n) = n$

Since $0 \leq M < n$, we can infer that $M = 0$.
Therefore, $C^{d'} \equiv M^{ed'} \equiv 0 \equiv M \pmod{n}$.
\end{proof}
\end{document}