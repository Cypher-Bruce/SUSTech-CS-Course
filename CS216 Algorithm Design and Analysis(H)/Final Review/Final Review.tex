\documentclass[a4paper,12pt]{article} 

% First, we usually want to set the margins of our document. For this we use the package geometry.
\usepackage[top = 2.5cm, bottom = 2.5cm, left = 2.5cm, right = 2.5cm]{geometry} 
\usepackage[T1]{fontenc}
\usepackage[utf8]{inputenc}

% The following two packages - multirow and booktabs - are needed to create nice looking tables.
\usepackage{multirow} % Multirow is for tables with multiple rows within one cell.
\usepackage{booktabs} % For even nicer tables.

% As we usually want to include some plots (.pdf files) we need a package for that.
\usepackage{graphicx} 

% The default setting of LaTeX is to indent new paragraphs. This is useful for articles. But not really nice for homework problem sets. The following command sets the indent to 0.
% \usepackage{setspace}
% \setlength{\parindent}{0in}
\usepackage{indentfirst}

% Package to place figures where you want them.
\usepackage{float}

% The fancyhdr package let's us create nice headers.
\usepackage{fancyhdr}

\usepackage{xcolor,amsmath,amsthm,algorithm2e,tikz,subcaption}
\RestyleAlgo{ruled}

\definecolor{myRed}{RGB}{211, 31, 17}
\definecolor{myOrange}{RGB}{244, 122, 0}
\definecolor{myLightTeal}{RGB}{98, 200, 211}
\definecolor{myDarkTeal}{RGB}{0, 113, 145}


% To make our document nice we want a header and number the pages in the footer.

\pagestyle{fancy} % With this command we can customize the header style.

\fancyhf{} % This makes sure we do not have other information in our header or footer.

\lhead{\footnotesize Algorithm Design and Analysis(H): Final Review}% \lhead puts text in the top left corner. \footnotesize sets our font to a smaller size.

%\rhead works just like \lhead (you can also use \chead)
\rhead{\footnotesize Mengxuan Wu} %<---- Fill in your lastnames.

% Similar commands work for the footer (\lfoot, \cfoot and \rfoot).
% We want to put our page number in the center.
\cfoot{\footnotesize \thepage} 

\begin{document}

\thispagestyle{empty} % This command disables the header on the first page. 

\begin{tabular}{p{15.5cm}}
{\large \bf Algorithm Design and Analysis(H)} \\
Southern University of Science and Technology \\ Mengxuan Wu \\ 12212006 \\
\hline
\\
\end{tabular}

\vspace*{0.3cm} %add some vertical space in between the line and our title.

\begin{center}
	{\Large \bf Final Review}
	\vspace{2mm}

	{\bf Mengxuan Wu}
		
\end{center}  

\vspace*{0.4cm} 

\section{Algorithm Analysis}

\subsection{Models of Computation}

\begin{enumerate}
	\item \textbf{Deterministic Turing Machine:} Run time is the number of steps. Memory is the number of cells. No random access to memory or any other random operations.
	\item \textbf{Word RAM:} Each memory location and input/output cell stores a $w$-bit integer/word. Run time is the number of primitive operations. Memory is the number of memories used.
\end{enumerate}

\subsection{Computational Complexity}

We measure the efficiency of an algorithm as a function of the input size, where time complexity is the number of computation steps and space complexity is the number of memory cells used.

\subsubsection{Desirable Scaling Properties}

If the input size is increased by a constant factor $c_1$, the run time should increase by at most a constant factor $c_2$. A polynomial-time algorithm, whose run time is bounded by $a n^b$ for some constants $a>0$ and $b>0$, obeys this property.

\subsubsection{Efficient Algorithms}

An algorithm is efficient if it runs in polynomial time. And this definition is insensitive to models of computation.

In real life, polynomial-time algorithms often have a small exponent and constant. Hence, they are efficient compared to brute-force algorithms.

\paragraph{Exceptions} For polynomial-time algorithms with a large exponent or constant, the algorithm may not be efficient in practice. Also, some exponential-time algorithms may be widely used because their worst-case instances are rare.

\subsubsection{Asymptotic Notation}

$T(n)$ is in $O(f(n))$ if there exist constants $c>0$ and $n_0>0$ such that $T(n) \leq c f(n)$ for all $n \geq n_0$.
$T(n)$ is in $\Omega(f(n))$ if there exist constants $c>0$ and $n_0>0$ such that $T(n) \geq c f(n)$ for all $n \geq n_0$.
$T(n)$ is in $\Theta(f(n))$ if $T(n)$ is both in $O(f(n))$ and in $\Omega(f(n))$.

\paragraph{Common Properties}

\begin{enumerate}
	\item Logarithms$<$Polynomials$<$Exponentials: For every $a>1$ and $c>0$, we have $\log_a n = O(n^c)$ and $n^c = O(a^n)$.
	\item Factorials: $n! = 2^{\Theta(n \log n)}$, i.e., $\log n! = \Theta(n \log n)$.
	\item Factorials$>$Exponentials: For every $r>1$, we have $r^n = O(n!)$.
\end{enumerate}

\section{Stable Matching}

\subsection{Definition}

A \textbf{perfect matching} is when everyone is matched monogamously. An unstable pair is a pair of people who prefer each other over their current partners. A \textbf{stable matching} is a perfect matching with no unstable pairs.

In some cases, a stable matching may not exist. While in other cases, there may be multiple stable matchings.

\subsection{Gale-Shapley Algorithm}

\begin{algorithm}[H]
	\caption{Gale-Shapley Algorithm}
	Initialize all people to be free\;
	\While{Some man is free and hasn't proposed to every woman}{
		Choose such man $m$\;
		Let $w$ be the highest ranked women in $m$'s preference list to whom $m$ has not yet proposed\;
		\If{$w$ is free}{
			$(m, w)$ become engaged\;
		}
		\ElseIf{$w$ prefers $m$ to her current partner $m'$}{
			$(m, w)$ become engaged and $(m', w)$ become free\;
		}
		\Else{
			$w$ rejects $m$\;
		}
	}
\end{algorithm}

\subsubsection{Termination}

\begin{proof}
Each man proposes to at most $n$ women. Since there are at most $n^2$ proposals, the algorithm terminates in $O(n^2)$ time.
\end{proof}


\subsubsection{Perfection}

\begin{proof}[Proof by Contradiction]
Assume someone is unmatched. Then, since the number of man and woman are equal, there must be another person of the opposite sex who is also unmatched. 

For the unmatched woman, she must have been proposed by any man, otherwise she would have been matched. But since the unmatched man has proposed to every woman, he must have proposed to the unmatched woman. 

Hence, we have a contradiction.
\end{proof}

\subsubsection{Stability}

\begin{proof}
Suppose $(m, w)$ is an unstable pair. 

\paragraph{Case 1} $m$ never proposed to $w$. Then, $m$ prefers his current partner to $w$. This contradicts the assumption that $(m, w)$ is unstable.

\paragraph{Case 2} $m$ proposed to $w$ and $w$ rejected $m$. Then, $w$ prefers her current partner to $m$. This contradicts the assumption that $(m, w)$ is unstable.
\end{proof}

\subsubsection{Other Properties}

Since multiple stable matchings may exist, which one is chosen by the algorithm?

We define a \textbf{valid partner} of a person as a partner that the person receives in some stable matching. The algorithm produces the following stable matchings:

\paragraph{Man Optimal} Each man receives his most preferred valid partner.

\begin{proof}[Proof by Contradiction]
Assume there exists some men who do not receive their most preferred valid partners. 

Let $m$ be the first men who do not receive his most preferred valid partner. Let $w$ be $m$'s most preferred valid partner. 

Since $m$ does not receive $w$, $w$ must prefer her current partner to $m$, denoted as $m'$. But since $m$ is the first men who do not receive his most preferred valid partner, $w$ must be the most preferred valid partner of $m'$.

Then, in the stable matching where $m$ is matched with $w$ (this match must exist since $w$ is $m$'s most preferred valid partner), $m'$ prefers $w$ to his current partner. And $w$ prefers $m'$ to her current partner. This contradicts the assumption that the matching is stable.
\end{proof}

\paragraph{Woman Pessimal} Each woman receives her least preferred valid partner.

\begin{proof}[Proof by Contradiction]
Assume there exists some woman who does not receive her least preferred valid partner.

Let $w$ be the woman who does not receive her least preferred valid partner. Let $m$ be $w$'s least preferred valid partner. Let $m'$ be $w$'s current partner. 

In some stable matching, $w$ is matched with $m$. Then, $m'$ prefers $w$ to his current partner (man optimal), and $w$ prefers $m'$ to $m$. This contradicts the assumption that the matching is stable.
\end{proof}

\subsection{Extensions}

\begin{enumerate}
	\item Some people declare unacceptable partners. 
	\item Unequal number on both sides.
	\item Limited positions (one to many).
\end{enumerate}

\section{Greedy Algorithms}

\subsection{Interval Scheduling}

\subsubsection{Definition}

Given a set of intervals, how to select the maximum number of non-overlapping intervals?

\subsubsection{Solution}

\paragraph{Greedy Approach} Select the compatible interval with the earliest finish time.

\subsection{Interval Partitioning}

\subsubsection{Definition}

Given a set of intervals, how to partition them into the minimum number of groups such that no two intervals in the same group overlap?

\subsubsection{Solution}

\paragraph{Greedy Approach} Consider the intervals in increasing order of their start times. Assign each interval to the group with the earliest finish time. If no group is available, create a new group.

\subsection{Minimizing Lateness}

\subsubsection{Definition}

Given a set of jobs with processing time and deadline, how to minimize the total lateness (the amount of time a job finishes after its deadline)?

\subsubsection{Solution}

\paragraph{Greedy Approach} Consider the jobs in increasing order of their deadlines. Schedule each job as early as possible.

\end{document}