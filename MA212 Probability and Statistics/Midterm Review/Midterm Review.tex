\documentclass[a4paper,12pt]{ctexart} 

% First, we usually want to set the margins of our document. For this we use the package geometry.
\usepackage[top = 2.5cm, bottom = 2.5cm, left = 2.5cm, right = 2.5cm]{geometry} 
\usepackage[T1]{fontenc}
\usepackage[utf8]{inputenc}

% The following two packages - multirow and booktabs - are needed to create nice looking tables.
\usepackage{multirow} % Multirow is for tables with multiple rows within one cell.
\usepackage{booktabs} % For even nicer tables.

% As we usually want to include some plots (.pdf files) we need a package for that.
\usepackage{graphicx} 

% The default setting of LaTeX is to indent new paragraphs. This is useful for articles. But not really nice for homework problem sets. The following command sets the indent to 0.
% \usepackage{setspace}
% \setlength{\parindent}{0in}

% Package to place figures where you want them.
\usepackage{float}

% The fancyhdr package let's us create nice headers.
\usepackage{fancyhdr}

\usepackage{amsmath,amsthm,mathabx}

% To make our document nice we want a header and number the pages in the footer.

\pagestyle{fancy} % With this command we can customize the header style.

\fancyhf{} % This makes sure we do not have other information in our header or footer.

\lhead{\footnotesize Probability and Statistics: Midterm Review}% \lhead puts text in the top left corner. \footnotesize sets our font to a smaller size.

%\rhead works just like \lhead (you can also use \chead)
\rhead{\footnotesize 吴梦轩} %<---- Fill in your lastnames.

% Similar commands work for the footer (\lfoot, \cfoot and \rfoot).
% We want to put our page number in the center.
\cfoot{\footnotesize \thepage} 

\begin{document}

\thispagestyle{empty} % This command disables the header on the first page. 

\begin{tabular}{p{15.5cm}}
{\large \bf Probability and Statistics} \\
Southern University of Science and Technology \\ 吴梦轩 \\ 12212006 \\
\hline
\\
\end{tabular}

\vspace*{0.3cm} %add some vertical space in between the line and our title.

\begin{center}
	{\Large \bf Midterm Review}
	\vspace{2mm}

	{\bf 吴梦轩}
		
\end{center}  

\vspace{0.4cm}

\section{随机变量}

\subsection{重要分布}

\subsubsection{二项分布}

二项分布的符号表示为$X\sim b(n,p)$,其频率函数为:
\begin{equation*}
	P\{X = x\} = C_n^xp^x(1-p)^{n-x} = \frac{n!}{x!(n-x)!}p^x(1-p)^{n-x}
\end{equation*}

意为:$n$次独立重复试验中,事件$A$发生$x$次的概率。

\textbf{多项分布:}

多项分布的符号表示为$X\sim m(n,p_1,p_2,\cdots,p_k)$,其频率函数为:
\begin{equation*}
	P\{X_1 = x_1,X_2 = x_2,\cdots,X_k = x_k\} = \frac{n!}{x_1!x_2!\cdots x_k!}p_1^{x_1}p_2^{x_2}\cdots p_k^{x_k}\ \left(\sum_{i=1}^kp_i = 1, \sum_{i=1}^kx_i = n\right)
\end{equation*}

意为:$n$次独立重复试验中,事件$A_1,A_2,\cdots,A_k$分别发生$x_1,x_2,\cdots,x_k$次的概率。

\subsubsection{泊松分布}

泊松分布的符号表示为$X\sim P(\lambda)$或$X\sim \pi(\lambda)$,其频率函数为:
\begin{equation*}
	P\{X = x\} = \frac{\lambda^x}{x!}e^{-\lambda}
\end{equation*}

意为:单位时间(或单位面积)内随机事件发生$x$次的概率。

\textbf{泊松定理:}

设$X\sim b(n,p)$,当$n\rightarrow\infty,p\rightarrow0$,使得$np=\lambda$不变,则有:
\begin{equation*}
	\lim_{n\rightarrow\infty}P\{X = x\} = \lim_{n\rightarrow\infty}C_n^xp^x(1-p)^{n-x} = \frac{\lambda^x}{x!}e^{-\lambda}
\end{equation*}

\textbf{泊松流:}

记在时间$(0,t]$内随机事件发生的次数为$N(t)$,则$N(t)\sim P(\lambda t)$,称$N(t)$为泊松流。
\begin{equation*}
	P\{N(t) = n\} = \frac{(\lambda t)^n}{n!}e^{-\lambda t}
\end{equation*}

\subsubsection{均匀分布}

均匀分布的符号表示为$X\sim U(a,b)$,其频率函数为:
\begin{equation*}
	P\{X = x\} = \begin{cases}
		\frac{1}{b-a}, & a<x<b \\
		0, & \text{otherwise}
	\end{cases}
\end{equation*}

其分布函数为:
\begin{equation*}
	F(x) = \begin{cases}
		0, & x\leq a \\
		\frac{x-a}{b-a}, & a<x<b \\
		1, & x\geq b
	\end{cases}
\end{equation*}

\subsubsection{正态分布}

正态分布的符号表示为$X\sim N(\mu,\sigma^2)$,其频率函数为:
\begin{equation*}
	P\{X = x\} = \frac{1}{\sqrt{2\pi}\sigma}e^{-\frac{(x-\mu)^2}{2\sigma^2}}
\end{equation*}

特别的,当$\mu = 0,\sigma = 1$时,称为标准正态分布,记为$X\sim N(0,1)$。
标准正态分布的分布函数记为$\Phi(x)$,其频率函数记为$\varphi(x)$。

\textbf{分布函数的积分过程:}
\begin{align*}
	F(+\infty) &= \int_{-\infty}^{+\infty}\frac{1}{\sqrt{2\pi}\sigma}e^{-\frac{(x-\mu)^2}{2\sigma^2}}\mathrm{d}x \\
	&= \frac{1}{\sqrt{2\pi}\sigma} \int_{-\infty}^{+\infty}e^{-\frac{(x-\mu)^2}{2\sigma^2}}\mathrm{d}x
\end{align*}

此时令$t = \frac{x-\mu}{\sigma}$,则有:
\begin{align*}
	F(+\infty) &= \frac{1}{\sqrt{2\pi}\sigma} \int_{-\infty}^{+\infty}e^{-\frac{(x-\mu)^2}{2\sigma^2}}\mathrm{d}x \\
	&= \frac{1}{\sqrt{2\pi}} \int_{-\infty}^{+\infty}e^{-\frac{t^2}{2}}\mathrm{d}t \\
	&= \frac{1}{\sqrt{2\pi}} \sqrt{2\pi} \\
	&= 1
\end{align*}

\textbf{规范化:}
证明若$X \sim N(\mu,\sigma^2)$,则$Z = \frac{X-\mu}{\sigma} \sim N(0,1)$。
\begin{align*}
	F_Z(z) &= P\{Z \leq z\} \\
	&= P\left\{\frac{X-\mu}{\sigma} \leq z\right\} \\
	&= P\{X \leq \sigma z + \mu\} \\
	&= F_X(\sigma z + \mu) \\
	&= \frac{1}{\sqrt{2\pi}\sigma} \int_{-\infty}^{\sigma z + \mu}e^{-\frac{(x-\mu)^2}{2\sigma^2}}\mathrm{d}x
\end{align*}

此时令$t = \frac{x-\mu}{\sigma}$,则有:
\begin{align*}
	F_Z(z) &= \frac{1}{\sqrt{2\pi}\sigma} \int_{-\infty}^{\sigma z + \mu}e^{-\frac{(x-\mu)^2}{2\sigma^2}}\mathrm{d}x \\
	&= \frac{1}{\sqrt{2\pi}} \int_{-\infty}^z e^{-\frac{t^2}{2}}\mathrm{d}t \\
	&= \Phi(z)
\end{align*}

\subsubsection{指数分布}

指数分布的符号表示为$X\sim EXP(\lambda)$,其频率函数为:
\begin{equation*}
	P\{X = x\} = \begin{cases}
		\lambda e^{-\lambda x}, & x\geq 0 \\
		0, & x<0
	\end{cases}
\end{equation*}

其分布函数为:
\begin{equation*}
	F(x) = \begin{cases}
		1-e^{-\lambda x}, & x\geq 0 \\
		0, & x<0
	\end{cases}
\end{equation*}

\textbf{指数分布与泊松流:}

设$N(t)$为泊松流,$N(t)\sim P(\lambda t)$。
记$X$为第一个事件发生的时间,则$P\{X > t\} = P\{N(t) = 0\} = e^{-\lambda t}$。
因此$X\sim EXP(\lambda)$。

\textbf{指数分布的无记忆性:}

设$X\sim EXP(\lambda)$,则对任意$s,t>0$,有:
\begin{align*}
	P\{X > s+t|X > s\} &= \frac{P\{X > s+t,X > s\}}{P\{X > s\}} \\
	&= \frac{P\{X > s+t\}}{P\{X > s\}} \\
	&= \frac{e^{-\lambda(s+t)}}{e^{-\lambda s}} \\
	&= e^{-\lambda t} \\
	&= P\{X > t\}
\end{align*}

\subsection{频率(密度)函数与分布函数}

\textbf{频率函数}的基本性质(本质特征):
\begin{itemize}
	\item $P\{X = x\} \geq 0$
	\item $\sum_{x\in X}P\{X = x\} = 1$
\end{itemize}

满足以上两个性质的数列必定是一个离散型随机变量的频率函数。

\textbf{密度函数}的基本性质(本质特征):
\begin{itemize}
	\item $f(x) \geq 0$
	\item $\int_{-\infty}^{+\infty}f(x)\mathrm{d}x = 1$
\end{itemize}

满足以上两个性质的函数必定是一个连续型随机变量的密度函数。

\textbf{分布函数}的基本性质(本质特征):
\begin{itemize}
	\item $F(x)$是一个单调不减函数
	\item $\lim_{x\rightarrow-\infty}F(x) = 0,\lim_{x\rightarrow+\infty}F(x) = 1$
	\item $F(x)$是右连续的,即$\lim_{x\rightarrow x_0^+}F(x) = F(x_0)$
\end{itemize}

满足以上三个性质的函数必定是一个随机变量的分布函数。

\subsection{分位数}

\textbf{分位数}:
设$X$是一个连续型随机变量,$F(x)$是其分布函数,对于$0<p<1$,若实数$x_p$使得$F(x_p) = p$,则称$x_p$为$X$的$p$分位数。
特殊的,当$p = 0.5$时,称为中位数。
当$p = 0.25$时,称为下四分之一位数。
当$p = 0.75$时,称为上四分之一位数。

\subsection{随机变量的函数的分布}

令$Y = g(X)$,当$Y$是严格单调函数时,有:
\begin{equation*}
	f_Y(y) = f_X(g^{-1}(y))\left|\frac{\mathrm{d}g^{-1}(y)}{\mathrm{d}y}\right|
\end{equation*}

正态分布的线性组合以及线性函数的分布仍然是正态分布。
假设有$n$个随机变量$X_1,X_2,\cdots,X_n$,且$X_i\sim N(\mu_i,\sigma_i^2)$,则有:
\begin{equation*}
	\sum_{i=1}^n a_i X_i + b \sim N\left(\sum_{i=1}^n a_i\mu_i + b, \sum_{i=1}^n a_i^2\sigma_i^2\right)
\end{equation*}

\subsection{二维随机变量}

\subsubsection{联合分布函数与边缘分布函数}

根据Farlie-Morgenstern定理,对于给定的两个一维随机变量$X,Y$,可以证明只要$|a| \leq 1$,就有:
\begin{equation*}
	H(x,y) = F(x)G(y) {1 + a[1 - F(x)][1 - G(y)]}
\end{equation*}
是二元连续型分布函数。

换言之,给定边际分布,可以构造出无限多个联合分布。

\subsubsection{连接函数}

将使得边缘分布为均匀分布的联合分布函数称为连接函数。
定义联合分布为:
\begin{equation*}
	F_{XY}(x,y) = C(F_X(x),F_Y(y))
\end{equation*}

定义密度函数为:
\begin{equation*}
	f_{XY}(x,y) = c(F_X(x),F_Y(y)) f_X(x) f_Y(y)
\end{equation*}

\subsubsection{二维正态分布}

二维正态分布记为$X\sim N(\mu_1,\mu_2,\sigma_1^2,\sigma_2^2,\rho)$,其联合密度函数为:
\begin{equation*}
	f(x,y) = \frac{1}{2\pi\sigma_1\sigma_2\sqrt{1-\rho^2}}\exp\left\{-\frac{1}{2(1-\rho^2)}\left[\frac{(x-\mu_1)^2}{\sigma_1^2} - 2\rho\frac{(x-\mu_1)(y-\mu_2)}{\sigma_1\sigma_2} + \frac{(y-\mu_2)^2}{\sigma_2^2}\right]\right\}
\end{equation*}

对于其边际分布,有:
\begin{equation*}
	X\sim N(\mu_1,\sigma_1^2),\ Y\sim N(\mu_2,\sigma_2^2)
\end{equation*}

证明如下:
\begin{align*}
	f_X(x) =& \int_{-\infty}^{+\infty}f(x,y)\mathrm{d}y \\
	=& \int_{-\infty}^{+\infty}\exp\left\{-\frac{1}{2(1-\rho^2)}\left[\frac{(x-\mu_1)^2}{\sigma_1^2} - 2\rho\frac{(x-\mu_1)(y-\mu_2)}{\sigma_1\sigma_2} + \frac{(y-\mu_2)^2}{\sigma_2^2}\right]\right\}\mathrm{d}y \\
	=& \frac{1}{2\pi\sigma_1\sigma_2\sqrt{1-\rho^2}} \exp\left\{-\frac{(x-\mu_1)^2}{2\sigma_1^2}\right\} \cdot \\
	 & \int_{-\infty}^{+\infty}\exp\left\{-\frac{1}{2(1-\rho^2)}\left[\frac{(y-\mu_2)^2}{\sigma_2^2} - 2\rho\frac{(x-\mu_1)(y-\mu_2)}{\sigma_1\sigma_2} + p^2 \frac{(x-\mu_1)^2}{\sigma_1^2}\right]\right\}\mathrm{d}y \\
	=& \frac{1}{2\pi\sigma_1\sigma_2\sqrt{1-\rho^2}} \exp\left\{-\frac{(x-\mu_1)^2}{2\sigma_1^2}\right\} \cdot \\
	 & \int_{-\infty}^{+\infty}\exp\left\{-\frac{1}{2(1-\rho^2)}\left[\frac{y-\mu_2}{\sigma_2} - \rho \frac{x-\mu_1}{\sigma_1}\right]^2\right\}\mathrm{d}y
\end{align*}

令$t = \frac{1}{\sqrt{1-\rho^2}}\left(\frac{y-\mu_2}{\sigma_2} - \rho \frac{x-\mu_1}{\sigma_1}\right)$,则有:
\begin{align*}
	f_X(x) =& \frac{1}{2\pi\sigma_1\sigma_2\sqrt{1-\rho^2}} \exp\left\{-\frac{(x-\mu_1)^2}{2\sigma_1^2}\right\} \cdot \\
	 & \int_{-\infty}^{+\infty}\exp\left\{-\frac{t^2}{2}\right\} \frac{1}{\sigma_2 \sqrt{1 - \rho^2}}\mathrm{d}t \\
	=& \frac{1}{\sqrt{2\pi}\sigma_1} \exp\left\{-\frac{(x-\mu_1)^2}{2\sigma_1^2}\right\} \\
	=& N(\mu_1,\sigma_1^2)
\end{align*}

注意:边际分布为正态分布时,联合分布不一定为二维正态分布。

\subsubsection{二维独立性}

设$X,Y$为二维随机变量,若对于任意$x,y$,有:
\begin{equation*}
	F(x,y) = F_X(x)F_Y(y)
\end{equation*}

则称$X,Y$相互独立。

对于离散型随机变量,有:
\begin{equation*}
	P\{X = x,Y = y\} = P\{X = x\}P\{Y = y\}
\end{equation*}

对于连续型随机变量,有:
\begin{equation*}
	f(x,y) = f_X(x)f_Y(y)
\end{equation*}

特别的,二维正态分布的独立性等价于$\rho = 0$。

两组独立的数据$(X_1, X_2, \cdots, X_n)$和$(Y_1, Y_2, \cdots, Y_m)$,其函数$h(X_1, X_2, \cdots, X_n)$和$g(Y_1, Y_2, \cdots, Y_m)$相互独立。

\subsubsection{条件分布}

设$X,Y$为二维随机变量,$f(x,y)$为其联合密度函数,$f_X(x)$为$X$的边际密度函数,若$f_X(x) > 0$,则称:
\begin{equation*}
	f_{Y|X}(y|x) = \frac{f(x,y)}{f_X(x)}
\end{equation*}

为$Y$在$X = x$的条件密度函数。

二维随机变量的全概率公式:
\begin{equation*}
	f_Y(y) = \int_{-\infty}^{+\infty}f_{Y|X}(y|x)f_X(x)\mathrm{d}x
\end{equation*}

特别的,对于二维正态分布,指定$X$的条件下$Y$的条件分布仍然是正态分布。
准确的说,若$(X,Y) \sim N(\mu_1,\mu_2,\sigma_1^2,\sigma_2^2,\rho)$,则有:
\begin{equation*}
	f_{Y|X}(y|x) \sim N\left(\mu_2 + \rho\frac{\sigma_2}{\sigma_1}(x-\mu_1),\sigma_2^2(1-\rho^2)\right)
\end{equation*}

\subsubsection{联合分布随机变量的函数}

设$X,Y$为二维随机变量,$Z = f(X,Y)$,则有:
\begin{equation*}
	F_Z(z) = P\{Z \leq z\} = P\{f(X,Y) \leq z\} = \iint_{f(x,y) \leq z}f(x,y)\mathrm{d}x\mathrm{d}y
\end{equation*}

\textbf{$Z = X + Y$的分布:}

\begin{align*}
	P\{Z \leq z\} &= P\{X + Y \leq z\} \\
	&= P\{X \leq z - Y\} \\
	&= \int_{-\infty}^{+\infty}\int_{-\infty}^{z-y}f(x,y)\mathrm{d}x\mathrm{d}y
\end{align*}

令$x = u - y$,则有:
\begin{align*}
	P\{Z \leq z\} &= \int_{-\infty}^{+\infty} \int_{-\infty}^{z} f(u-y,y)\mathrm{d}u\mathrm{d}y \\
	f_Z(z) &= \int_{-\infty}^{\infty} f(z-y,y)\mathrm{d}y \\
	&= \int_{-\infty}^{\infty} f(x,z-x)\mathrm{d}x
\end{align*}

特别的,当$X,Y$相互独立时,有:
\begin{equation*}
	f_Z(z) = \int_{-\infty}^{\infty} f_X(x)f_Y(z-x)\mathrm{d}x = \int_{-\infty}^{\infty} f_X(z-y)f_Y(y)\mathrm{d}y
\end{equation*}

特别的,当$X \sim P(\lambda_1), Y \sim P(\lambda_2)$时,$Z = X + Y \sim P(\lambda_1 + \lambda_2)$。

\textbf{$Z = \frac{X}{Y}$的分布:}

同理,通过换元法,有:
\begin{equation*}
	f_Z(z) = \int_{-\infty}^{\infty} |y|f_X(zy,y)\mathrm{d}y = \int_{-\infty}^{\infty} |x|f_X(x,xz)\mathrm{d}x
\end{equation*}

\textbf{两个随机变量变换的分布:}

如果两个随机变量的联合分布为二维正态分布,则他们的非奇异线性变换的分布仍然是二维正态分布。

\subsection{极值}

\begin{align*}
	F_{\text{max}}(z) &= P\{X \leq z,Y \leq z\} \\
	&= P\{X \leq z\}P\{Y \leq z\} \\
	&= F_X(z)F_Y(z) \\
	F_{\text{min}}(z) &= 1 - P\{X > z,Y > z\} \\
	&= 1 - P\{X > z\}P\{Y > z\} \\
	&= 1 - [1 - F_X(z)][1 - F_Y(z)]
\end{align*}

特别的,如果$X \sim EXP(\lambda), Y \sim EXP(\mu)$,则$Z = \min\{X,Y\} \sim EXP(\lambda + \mu)$。

\subsection{顺序统计量}

$X_{(k)}$的密度函数为:
\begin{equation*}
	f_{X_{(k)}}(x) = \frac{n!}{(k-1)!(n-k)!}[F(x)]^{k-1}[1-F(x)]^{n-k}f(x)
\end{equation*}

$(X_{(1)},X_{(n)})$的联合密度函数为:
\begin{equation*}
	f_{X_{(1)},X_{(n)}}(x,y) = n(n-1)[F(y) - F(x)]^{n-2}f(x)f(y)
\end{equation*}

\section{古典概型}

\subsection{集合}

试验的样本点记为$\omega$,样本空间记为$\Omega$,样本空间的子集称为事件。

当事件$A$与事件$B$不可能同时发生时,称事件$A$与事件$B$互不相容,或互斥,或互不相交,记为$A \cap B = \emptyset$。

加法公式:
\begin{equation*}
	P(A\cup B) = P(A) + P(B) - P(AB)
\end{equation*}

\subsubsection{计数方法}

选排列:
从$n$个不同元素中任取$m$个元素,按照一定的顺序排成一列,称为从$n$个不同元素中选取$m$个元素的排列,记为$A_n^m = \frac{n!}{(n-m)!}$。
特别的,当$m = n$时,称为全排列,记为$A_n^n = n!$。

组合:
从$n$个不同元素中任取$m$个元素,不考虑顺序,称为从$n$个不同元素中选取$m$个元素的组合,记为$C_n^m = \frac{A_n^m}{m!} = \frac{n!}{m!(n-m)!}$。
特别的,将$n$个不同元素分成$r$组,使得第$i$组有$n_i$个元素,且$\sum_{i=1}^rn_i = n$,则称为将$n$个不同元素分成$r$组的组合,记为$C_n^{n_1,n_2,\cdots,n_r} = \frac{n!}{n_1!n_2!\cdots n_r!}$。

将$n$个相同的球放入$m$个不同的盒子中,每个盒子中至少有一个球的方法数为$C_{n-1}^{m-1}$。(相当于在$n-1$个间隔中选取$m-1$个间隔放入隔板)
将$n$个相同的球放入$m$个不同的盒子中,盒子可以为空的方法数为$C_{n+m-1}^{m-1}$。(相当于将$n+m$个相同的球放入$m$个不同的盒子中,每个盒子中至少有一个球)

\subsection{全概率公式与贝叶斯公式}

全概率公式:
设$B_1,B_2,\cdots,B_n$为样本空间$\Omega$的一个划分,即$B_i\cap B_j = \emptyset(i\neq j),\bigcup_{i=1}^nB_i = \Omega$,则对任一事件$A$,有:
\begin{equation*}
	P(A) = \sum_{i=1}^nP(A|B_i)P(B_i)
\end{equation*}

贝叶斯公式:
设$B_1,B_2,\cdots,B_n$为样本空间$\Omega$的一个划分,即$B_i\cap B_j = \emptyset(i\neq j),\bigcup_{i=1}^nB_i = \Omega$,则对任一事件$A$,有:
\begin{equation*}
	P(B_i|A) = \frac{P(A|B_i)P(B_i)}{\sum_{j=1}^nP(A|B_j)P(B_j)}
\end{equation*}

\subsection{独立性}

设$A,B$为两事件,若$P(AB) = P(A)P(B)$,则称事件$A$与事件$B$相互独立。

独立不同于互不相容,独立是指两事件发生的概率互不影响,互不相容是指两事件不能同时发生。

两两独立不能推出多个事件相互独立。独立也没有传递性,即$A,B$独立,$B,C$独立,不能推出$A,C$独立。

\end{document}