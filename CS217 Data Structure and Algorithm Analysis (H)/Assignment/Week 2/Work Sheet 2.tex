
\documentclass{article}
\usepackage{indentfirst, amsmath, amsthm, graphicx}


\title{Exercise Sheet 2}
\author{Mengxuan Wu}
\date{\today}

\begin{document}

\maketitle

\section*{Question 2}
\subsection*{Question 2.1}
\begin{align*}
3n^2 + 5n - 2 = \Theta (n^2) \qquad &c_1 = 2 \quad c_2 = 4 \quad n_0 = 5\\
42 = \Theta (1) \qquad &c_1 = 41 \quad c_2 = 43 \quad n_0 = 1\\ 
4n^2\cdot(1+\log n)-2n^2 = \Theta (n^2\log n) \qquad &c_1 = 3 \quad c_2 = 5 \quad n_0 = 4\\
\end{align*}

\subsection*{Question 2.2}
\begin{center}
    \resizebox*{0.7\linewidth}{!}{
        \begin{tabular}{|cc|c|c|c|c|c|}
            \hline
            $f(n)$ & $g(n)$ & $O$ & $o$ & $\Omega$ & $\omega$ & $\Theta$\\ \hline \hline
            $\log n$ & $\sqrt{n}$ & yes & yes & no & no & no\\ \hline
            $n$ & $\sqrt{n}$ & no & no & yes & yes & no\\ \hline
            $n$ & $n\log n$ & yes & yes & no & no & no\\ \hline
            $n^2$ & $n^2+(\log n)^3$ & yes & no & yes & no & yes\\ \hline
            $2^n$ & $n^3$ & no & no & yes & yes & no\\ \hline
            $2^{n/2}$ & $2^n$ & yes & yes & no & no & no\\ \hline
            $\log_2 n$ & $\log_{10} n$ & yes & no & yes & no & yes\\ \hline
        \end{tabular}}
\end{center}

\subsection*{Question 2.3}
For algorithm $A$, line 1 is executed once and line 4, 5 and 6 are executed $n^2-2n$ times each.
The number of foo operation is $3n^2-6n+1=\Theta(n^2)$, as $2n^2 \leq 3n^2-6n+1 \leq 4n^2$ for all $n \geq 6$.

For algorithm $B$, line 1 is executed once, line 3 is executed $n$ times and line 5 and 6 are executed $n/2$ times each.
The number of foo operation is $2n+1=\Theta(n)$, as $n \leq 2n+1 \leq 3n$ for all $n \geq 1$.

For algorithm $C$, line 1 and 6 are executed once each, line 4 is executed $n(n+1)/2$ times and line 5 is executed $n$ times.
The number of foo operation is $\frac{1}{2}n^2+\frac{3}{2}n+2=\Theta(n^2)$, as $\frac{1}{2}n^2 \leq \frac{1}{2}n^2+\frac{3}{2}n+2 \leq n^2$ for all $n \geq 4$.

\subsection*{Question 2.4}
1. True. 
\begin{proof}
    For all $f(n) \in O(\sqrt{n})$, there exists some $c, n_0 > 0$ that satisfy $0 \leq f(n) \leq c\sqrt{n}$ for all $n \geq n_0$. 
    Since $0 \leq c\sqrt{n} \leq cn$ holds for all positive integer $n$, we can infer that $0 \leq f(n) \leq c\sqrt{n} \leq cn$ for all $n \geq n_0$, with the same $c$ and $n_0$.
    Hence, we have $f(n) \in O(n)$.
\end{proof}

2. False.
\begin{proof}
    Since $n = o(n^2)$ holds, we can infer that $n+n=2n=\omega(n)$. 
    However, $\lim_{n \rightarrow \infty}\frac{2n}{n}=2\ne\infty$. 
    There is a contradiction.
\end{proof}

3. True.
\begin{proof}
    Fisrtly, we will proof $3n\log n + O(n) = O(n\log n)$. 
    Since there exists some $c_0, n_0 > 0$ that satisfy $0 \leq 3n\log n + O(n) \leq 3n\log n + c_0n$ for all $n \geq n_0$, we can deduce that $0 \leq 3n\log n + O(n) \leq 3n\log n + c_0n < 3n\log n + c_0n\log n = (3+c_0)n\log n$.
    Hence, we have $3n\log n + O(n) = O(n\log n)$, with $c_1 = c_0 + 3$ and $ n_1 = n_0$.

    Secondly, we will proof $3n\log n + O(n) = \Omega(n\log n)$. 
    Since $O(n)$ has a non-negative value, we can infer that $0 \leq 3n\log n \leq 3n\log n + O(n)$.
    Hence, we have $3n\log n + O(n) = \Omega(n\log n)$, with $c_2 = 3$ and $ n_2 = n_0$.

    Finally, with two conclusions above, we can proof $3n\log n + O(n) = \Theta(n\log n)$.
\end{proof}

4. The statement "The running time of Algorithm A is at least $O(n^2)$" is meaningless. 
Because notation $O$ means "at most" already, it contradicts "at least" before it.

\subsection*{Question 2.5}
\begin{center}
    \resizebox*{\textwidth}{!}{
        \begin{tabular}{lc}
            \hline
            \textsc{Matrix-Multiply$(A, B)$} & Runtime(in one iteration)\\ \hline
            1: \textbf{for} $i=1$ to $n$ \textbf{do} & $n+1=\Theta(n)$\\
            2: \qquad \textbf{for} $j=1$ to $n$ \textbf{do} & $n+1 = \Theta(n)$\\
            3: \qquad \qquad $C[i,j]=0$ & $1=\Theta(1)$\\
            4: \qquad \qquad \textbf{for} $k=1$ to $n$ \textbf{do} & $n+1=\Theta(n)$\\ 
            5: \qquad \qquad \qquad $C[i,j]=C[i,j]+A[i,k]\cdot B[k,j]$ & $1=\Theta(1)$\\
            6: \textbf{return} $C$ & $1=\Theta(1)$\\
            \hline
        \end{tabular}}
\end{center}

The total runtime of \textsc{Matrix-Multiply} is

\begin{equation*}
    \Theta(n) \cdot \Theta(n) \cdot (\Theta(1) + \Theta(n) \cdot \Theta(1)) + \Theta(1) = \Theta(n^3)
\end{equation*}

\end{document}