\documentclass[a4paper,12pt]{ctexart} 

% First, we usually want to set the margins of our document. For this we use the package geometry.
\usepackage[top = 2cm, bottom = 2cm, left = 2.5cm, right = 2.5cm]{geometry} 

% The following two packages - multirow and booktabs - are needed to create nice looking tables.
\usepackage{multirow} % Multirow is for tables with multiple rows within one cell.
\usepackage{booktabs} % For even nicer tables.

% As we usually want to include some plots (.pdf files) we need a package for that.
\usepackage{graphicx} 

% The default setting of LaTeX is to indent new paragraphs. This is useful for articles. But not really nice for homework problem sets. The following command sets the indent to 0.
% \usepackage{setspace}
% \setlength{\parindent}{0in}
\usepackage{indentfirst}

% Package to place figures where you want them.
\usepackage{float}

\usepackage{amsmath,amsthm,mathabx}

\begin{document}

\begin{center}
	{\Large \bf Assignment 1}
\end{center}  

\subsection*{作为工程师的建议}

\subsubsection*{组建RAID阵列:防止物理破坏}

组建RAID阵列可以有效防止物理破坏导致的数据丢失,并提高服务器的读写速度。
如果组建RAID10阵列,服务器可以在多块硬盘损坏的情况下不丢失数据,且采用热备份的方式,不会影响服务器的正常运行。


\subsubsection*{备份数据库:防止数据丢失}

公司应对数据库保持一定的备份频率,在出现数据丢失的情况下,可以回滚到最近一次备份的状态。
为保证数据安全,应将备份的数据存储在不同的地点,且将访问不同备份的权限分配给不同的人,防止单个人员恶意篡改备份的数据。

考虑到公司的规模较小,传统的两地三中心备份方案不适用。
但是可以通过购买云服务器的方式实现热备份或者冷备份。

\subsubsection*{数据审计:防止数据篡改}

通过设置trigger,可以在每次对数据库进行修改时,将修改的内容记录在日志中。
如果发现数据库中的数据被恶意篡改,可以通过日志追踪到修改的时间和修改的内容,从而找到破坏数据库的人。

\subsubsection*{输入检查:防止SQL注入攻击}

在设计数据库输入时,使用prepared statement,可以防止SQL注入攻击篡改或者删除数据库中的数据。
使用procedure,可以提前对输入的数据进行检查,同样可以防止恶意操作。

\subsubsection*{身份验证:防止密码泄露导致的恶意操作}

对有权修改数据库的用户,应在操作前进行多重身份验证。
例如通过离线应用程序随机生成并不断更新验证码(例如 Google Authenticator),可以保证即使高权限人员密码泄露,攻击者也无法进行恶意操作。

\subsubsection*{成本估计}

\begin{center}
	\resizebox{\linewidth}{!}{
		\begin{tabular}{lll}
			\toprule
			\textbf{项目} & \textbf{成本} & \textbf{说明} \\ \midrule
			本地备份/RAID10阵列 & 0.29\textyen/GB & 以西部数据4TB黑盘为例 \\
			数据库云备份(冷备份) & 0.12\textyen/GB/月(储存费用)+0.75\textyen/GB(备份费用) & 以阿里云DBS服务为例 \\
			数据库云备份(热备份) & 可按需购买计算节点及储存空间 & 以阿里云PolarDB PostgreSQL版为例 \\
			数据审计 & 0.29\textyen/GB & 同本地备份/RAID10阵列 \\
			其余 & 0.00\textyen & 无需额外成本 \\
			\bottomrule
		\end{tabular}
	}
\end{center}

\newpage

\subsection*{作为CTO的建议}

\subsubsection*{权限管理:防止低权限用户进行恶意操作}

在软件层面,应对用户进行权限管理。
例如负责前端的工程师不应该拥有修改数据库的权限,而只能通过调用procedure的方式修改数据库,如需要修改数据库,则需要向负责后端的工程师提出申请。
负责维护数据库的工程师应该拥有最高权限,但是应该对其进行严格的审计。

已经离职的人员的授权应该被删除,以防止公司以外的人利用账号进行恶意操作。

\subsubsection*{完善数据库规范}

公司可以出台数据库规范或遵守已有的数据库规范,对数据库的设计、维护、备份等方面的操作方法提出要求,并规律性地通过审计来检查规范的执行情况。
对于违反规范的人员,应该进行相应的处罚。

同时,应该对有权限修改数据库的人员进行普法教育,提高其安全意识。

\subsubsection*{完善灾备计划}

公司应该制定完善的灾备计划。
应规律进行应急演练,模拟数据库被破坏的情况,以保证在真正发生灾难时,公司能够快速恢复。

\subsubsection*{成本估计}

\begin{center}
	\begin{tabular}{lll}
		\toprule
		\textbf{项目} & \textbf{成本} & \textbf{说明} \\ \midrule
		权限管理 & 开发权限管理软件成本 & 无需额外成本 \\
		数据库规范 & 0.00\textyen & 无需额外成本 \\
		灾备计划 & 0.00\textyen & 无需额外成本 \\
		\bottomrule
	\end{tabular}
\end{center}


\end{document}